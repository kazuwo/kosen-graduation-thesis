\documentclass[11pt,a4paper]{jsarticle}
%
\usepackage{amsmath,amssymb}
\usepackage{bm}
\usepackage{graphicx}
\usepackage{ascmac}
%
\setlength{\textwidth}{\fullwidth}
\setlength{\textheight}{40\baselineskip}
\addtolength{\textheight}{\topskip}
\setlength{\voffset}{-0.2in}
\setlength{\topmargin}{0pt}
\setlength{\headheight}{0pt}
\setlength{\headsep}{0pt}
%
\newcommand{\divergence}{\mathrm{div}\,}  %ダイバージェンス
\newcommand{\grad}{\mathrm{grad}\,}  %グラディエント
\newcommand{\rot}{\mathrm{rot}\,}  %ローテーション
%
\title{xeeシステムを用いた授業評価アンケートの開発}
\author{溝 大貴}
\begin{document}
\maketitle
%
%
\section{はじめに}
 本研究の目的は、web上で実施する授業評価アンケートシステムを開発することである。従来、鹿児島工業高等専門学校では紙媒体で授業評価アンケートを実施していた。紙媒体でアンケート調査をすると、多くのコストがかかる。例えば、アンケート用紙の代金や、一クラスにつき一人必要な試験監督への人件費などである。一方で、web上でアンケートを実施するとこれらの費用が掛からない。しかも、集計が楽な点や、場所に囚われな移転など、大きなメリットを得ることができる。
近年、授業評価アンケートに大きな注目が集まっている~的な内容でもいいかも
近年、社会選択理論をはじめとする集団の意思決定方法に注目が集まっている。選挙の結果を異なる集約方法で集計すると結果が変わるように、意見の集約方法は重要である。そのような中、本稿では鹿児島工業高等専門学校の授業評価アンケートに注目した。

\section{リサーチクエスチョン}
 本研究では鹿児島工業高等専門学校の授業評価アンケートの改善を目的とする。従来の授業評価アンケートは紙媒体のマークシート方式を採用していた。この手法は、紙媒体であるにもかかわらず、結果の読み取りをコンピュータで自動化できるという長所がある。しかし、マークシート方式には次のような短所がある。
\begin{itemize}
  \item 集めたデータの集計が面倒。 
  \item 試験監督が1クラスに1人必用。 
  \item 回答する時間と場所を拘束される。
  \item 貴重な紙資源を消費してしまう。
\end{itemize}
これは、現在の鹿児島工業高等専門学校の授業評価アンケートの欠点であるとも言える。この問題を解決るために、webアプリケーション上で授業評価アンケートを完結させる手法を提案する。webアプリケーションを用いて授業評価アンケートを実施することで、紙を1枚も使わず、回答する時間も場所も自由にアンケートに回答できる。また、試験監督は必要なくなり、データの集計も自動化することができる。
\section{先行研究}
web上でアンケートを実施するフレームワークとしては、Google者の「Google フォーム」や有限会社ディアイピィの「DIPSurvey-Free」などがある。これらは、ユーザーが簡単にwebアンケートを作成することを第一に考えたサービスである。そのため、アンケートはGUIを用いて作成する。具体的には、あらかじめ用意されていたテンプレートから目的に合った物を選択し、それらを連結してアンケートを作成する。
\section{ソフトウェアの開発}
\subsection{作製したシステムと既存のシステムの差}
本研究では、先行研究で見られた複数のテンプレートをGUI上で結合するアンケート作製方式を使用しない。なぜなら、インターフェースの種類や、データの集計方法に大きな制約があるからである。そこで、本研究ではxeeシステムを用いて授業評価アンケートを作成した。xeeシステムでは、Material UIを用いて自由にインターフェースを作ることができる。また、データの集計、配布方法もプログラミングを用いて自由に行うことができる。
\subsection{xeeシステムについて}
xeeシステムとは林(2016)が発表したオンライン経済実験の基盤システムである。xeeシステムはシステム基幹部分と、その配下にある実験群から構成される。xeeシステム基幹部分はErlang VM上で動作するElixirという言語で書かれたwebアプリケーションの基盤であるPhoenix Frameworkを利用して設計されている。そして、実験群はFacebook社が開発したReact.jsとGoogle Material Designが提供しているUIパーツであるMaterial UI、基幹部分と連携するためのElixirで構成される。
\subsection{xeeシステムに使われた技術}
\subsubsection{erlang}
Erlangは1987年ごろにスウェーデンの電話会社Ericssonで開発された関数型言語である。Erlangのランタイムシステムには、平行性、分散性、耐障害性のサポートが組み込まれている。そのため今では、電気通信、銀行業務、電子商取引など広い分野のシステムに組み込まれている。
\subsubsection{elixir}
ElixirはErlang VM上で動作する関数型言語である。これはErlangの使いづらさ(文字列処理が苦手)などを改変し、Ruby風に書けるようにした言語である。
\subsubsection{phoenix framework}
phoenix frameworkはElixirで書かれたwebフレームワークのことである。これはRailsやDjangoのようなフレームワークに影響を受けており、MVCパターンにのっとり開発をすることができる。
\subsubsection{React}
ReactはFacebook社が開発したJavaScriptライブラリである。これはwebアプリケーションのview部分を作ることに特化している。ReactはバーチャルDOMという概念を採用しており、JavaScriptでよりも高速に描画することができる。
\subsubsection{webpack}
webpackはモダンなJavaScriptアプリケーションのモジュールビルダである。これは、複数のファイル依存関係を反映して一つのファイルにまとめる働きをする。
\section{授業評価アンケートの開発}
\subsection{作製した授業評価アンケートの概要}
本研究で作製した授業評価アンケートは
\subsection{操作手順}
\subsection{ファイル構成}
\subsection{待機画面の説明}
\subsection{説明画面の説明}
\subsection{実験画面の説明}
\subsection{終了画面の説明}
\section{考察}
\section{今後の課題}
\section{まとめ}
\section{参考文献}
http://www.geocities.jp/m_hiroi/func/abcerl01.html
\section{謝辞}
%
%
\end{document}
