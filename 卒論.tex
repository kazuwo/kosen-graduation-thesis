\documentclass[11pt,a4paper]{jsarticle}
%
\usepackage{amsmath,amssymb}
\usepackage{bm}
\usepackage{graphicx}
\usepackage{ascmac}
%
\setlength{\textwidth}{\fullwidth}
\setlength{\textheight}{40\baselineskip}
\addtolength{\textheight}{\topskip}
\setlength{\voffset}{-0.2in}
\setlength{\topmargin}{0pt}
\setlength{\headheight}{0pt}
\setlength{\headsep}{0pt}
%
\newcommand{\divergence}{\mathrm{div}\,}  %ダイバージェンス
\newcommand{\grad}{\mathrm{grad}\,}  %グラディエント
\newcommand{\rot}{\mathrm{rot}\,}  %ローテーション
%
\title{xeeシステムを用いた授業評価アンケートの開発}
\author{溝 大貴}
\begin{document}
\maketitle
%
%
\section{はじめに}
 本研究の目的は、web上で実施する授業評価アンケートシステムを開発することである。従来、鹿児島工業高等専門学校では紙媒体で授業評価アンケートを実施していた。紙媒体でアンケート調査をすると、多くのコストがかかる。例えば、アンケート用紙の代金や、一クラスにつき一人必要な試験監督への人件費などである。一方で、web上でアンケートを実施するとこれらの費用が掛からない。しかも、集計が楽な点や、場所に囚われな移転など、大きなメリットを得ることができる。
近年、授業評価アンケートに大きな注目が集まっている~的な内容でもいいかも
近年、社会選択理論をはじめとする集団の意思決定方法に注目が集まっている。選挙の結果を異なる集約方法で集計すると結果が変わるように、意見の集約方法は重要である。そのような中、本稿では鹿児島工業高等専門学校の授業評価アンケートに注目した。

\section{リサーチクエスチョン}
 本研究では鹿児島工業高等専門学校の授業評価アンケートの改善を目的とする。従来の授業評価アンケートは紙媒体のマークシート方式を採用していた。この手法は、紙媒体であるにもかかわらず、結果の読み取りをコンピュータで自動化できるという長所がある。しかし、マークシート方式には次のような短所がある。
\begin{itemize}
  \item 集めたデータの集計が面倒。 
  \item 試験監督が1クラスに1人必用。 
  \item 回答する時間と場所を拘束される。
  \item 貴重な紙資源を消費してしまう。
\end{itemize}
これは、現在の鹿児島工業高等専門学校の授業評価アンケートの欠点であるとも言える。この問題を解決るために、webアプリケーション上で授業評価アンケートを完結させる手法を提案する。webアプリケーションを用いて授業評価アンケートを実施することで、紙を1枚も使わず、回答する時間も場所も自由にアンケートに回答できる。また、試験監督は必要なくなり、データの集計も自動化することができるようになる。
\section{先行研究}
web上でアンケートを実施するフレームワークとしては、Google者の「Google フォーム」や有限会社ディアイピィの「DIPSurvey-Free」などがある。これらは、ユーザーが簡単にwebアンケートを作成することを第一に考えたサービスである。そのため、アンケートはGUIを用いて作成する。具体的には、あらかじめ用意されていたテンプレートから目的に合った物を選択し、それらを連結してアンケートを作成する。このような方式は一見便利に見えるが、裏を返せば自分の自由にアンケートを作れないということである。この制約は、授業評価アンケートを作成する際に大きな障壁となり得る。そのため本研究では、大手企業が一般ユーザ向けに作成したアンケート作製フレームワークは使用しない。
\section{ソフトウェアの開発}
\subsection{作製したシステムと既存のシステムの差}
\subsection{xeeシステムについて}
\subsection{xeeシステムに使われた技術}
\subsubsection{erlang}
\subsubsection{elixir}
\subsubsection{jsx}
\subsubsection{webpack}
\section{授業評価アンケートの開発}
\subsection{作製した授業評価アンケートの概要}
\subsection{操作手順}
\subsection{ファイル構成}
\subsection{待機画面の説明}
\subsection{説明画面の説明}
\subsection{実験画面の説明}
\subsection{終了画面の説明}
\section{考察}
\section{今後の課題}
\section{まとめ}
\section{参考文献}
\section{謝辞}
%
%
\end{document}
