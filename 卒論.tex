\documentclass[11pt,a4paper]{jsarticle}
%
\usepackage{amsmath,amssymb}
\usepackage{bm}
\usepackage{graphicx}
\usepackage{ascmac}
%
\setlength{\textwidth}{\fullwidth}
\setlength{\textheight}{40\baselineskip}
\addtolength{\textheight}{\topskip}
\setlength{\voffset}{-0.2in}
\setlength{\topmargin}{0pt}
\setlength{\headheight}{0pt}
\setlength{\headsep}{0pt}
%
\newcommand{\divergence}{\mathrm{div}\,}  %ダイバージェンス
\newcommand{\grad}{\mathrm{grad}\,}  %グラディエント
\newcommand{\rot}{\mathrm{rot}\,}  %ローテーション
%
\title{xeeシステムを用いた授業評価アンケートの開発}
\author{溝 大貴}
\begin{document}
\maketitle
%
%
\section{はじめに}
 本研究の目的は、web上で実施する授業評価アンケートシステムを開発することである。現在、アンケートを実施するには2つの方法が考えられる。1つ目は紙媒体でのアンケートである。このアンケート方式では質問紙を配布し回答者が回答する。その質問方法として自由記入方式や間隔尺度、部分尺度などがある。特に尺度を用いて回答方式に一定の回答ルールを適用することで回答結果を統計的に集計することが出来る。紙媒体のアンケート方法の中で、最もコンピュータを用いた集計に適しているのがマークシート方式である。これは、回答済みの用紙をコンピュータで読み取ることが出来る。そのため、質問用紙からコンピュータにデータを移す作業が自動化出来る。
2つ目の方法はWEBページを用いたアンケートである。このアンケート方式では、回答者は指定されたWEBページでアンケートに回答する。このアンケートの質問方式は紙のアンケートと同じものを使用することが出来る。それに加えてGIFやCSSを使用することにより動画を挿入することも可能である。また、HTML5を使用することでドラックアンドドロップを用いた並び替えも可能である。回答したデータはサーバーに転送される。そして、司会者はサーバーに溜まったデータを統計的に分析する。
この2つの方法を比較すると、WEBを用いたアンケートのほうが質問する際の表現の幅が広くインタラクティブなUIを実装することが出来る事が分かる。インタラクティブとは、斎藤先生からいんようしましょ。また、紙資源の節約にもつながる。しかし、WEBアンケートは製作者にWEBの知識が無いと作成することが難しい。WEBアンケートは手軽さにおいて紙アンケートに劣る。

\section{リサーチクエスチョン}
本研究では、鹿児島工業高等専門学校で実施されていたマークシート方式の授業評価アンケートをWEBアンケートに変更し、よりインタラクティブなアンケートに改善することを目的とする。従来の鹿児島工業高等専門学校の授業評価アンケートは紙媒体のマークシート方式を採用していた。この手法ではインタラクティブなUIの実装は難しく無駄な紙資源を消費する。そこで、XEEシステムを用いてインタラクティブなUIを実装したWEBアンケートを作成した。

\section{先行研究}
web上でアンケートを実施するフレームワークとしては、Google者の「Google フォーム」や有限会社ディアイピィの「DIPSurvey-Free」などがある。これらは、ユーザーが簡単にwebアンケートを作成することを第一に考えたサービスである。そのため、アンケートはGUIを用いて作成する。具体的には、あらかじめ用意されていたテンプレートから目的に合った物を選択し、それらを連結してアンケートを作成する。
\section{ソフトウェアの開発}
\subsection{作製したシステムと既存のシステムの差}
本研究では、先行研究で見られた複数のテンプレートをGUI上で結合するアンケート作製方式を使用しない。なぜなら、インターフェースの種類や、データの集計方法に大きな制約があるからである。そこで、本研究ではxeeシステムを用いて授業評価アンケートを作成した。xeeシステムでは、Material UIを用いて自由にインターフェースを作ることができる。また、データの集計、配布方法もプログラミングを用いて自由に行うことができる。
\subsection{xeeシステムについて}
xeeシステムとは林(2016)が発表したオンライン経済実験の基盤システムである。xeeシステムはシステム基幹部分と、その配下にある実験群から構成される。xeeシステム基幹部分はErlang VM上で動作するElixirという言語で書かれたwebアプリケーションの基盤であるPhoenix Frameworkを利用して設計されている。そして、実験群はFacebook社が開発したReact.jsとGoogle Material Designが提供しているUIパーツであるMaterial UI、基幹部分と連携するためのElixirで構成される。
\subsection{xeeシステムに使われた技術}
\subsubsection{erlang}
Erlangは1987年ごろにスウェーデンの電話会社Ericssonで開発された関数型言語である。Erlangのランタイムシステムには、平行性、分散性、耐障害性のサポートが組み込まれている。そのため今では、電気通信、銀行業務、電子商取引など広い分野のシステムに組み込まれている。
\subsubsection{elixir}
ElixirはErlang VM上で動作する関数型言語である。これはErlangの使いづらさ(文字列処理が苦手)などを改変し、Ruby風に書けるようにした言語である。
\subsubsection{phoenix framework}
phoenix frameworkはElixirで書かれたwebフレームワークのことである。これはRailsやDjangoのようなフレームワークに影響を受けており、MVCパターンにのっとり開発をすることができる。
\subsubsection{React}
ReactはFacebook社が開発したJavaScriptライブラリである。これはwebアプリケーションのview部分を作ることに特化している。ReactはバーチャルDOMという概念を採用しており、JavaScriptでよりも高速に要素を描画することができる。
\subsubsection{webpack}
webpackはモダンなJavaScriptアプリケーションのモジュールビルダである。これは、複数のファイル依存関係を反映して一つのファイルにまとめる働きをする。
\section{授業評価アンケートの開発}
\subsection{作製した授業評価アンケートの概要}
本研究で作製した授業評価アンケートは参加者と司会役の2種類の人で行う。参加者とは授業を評価する学生たちのことであり、30~45人ほどを想定している。また、司会者とはクラスごとに授業評価アンケートを行う人である。司会者は参加者の進行具合に合わせてアンケートシステムを操作する。参加者が評価を終えて、司会者がアンケートシステムを操作すると、データが自動で集計される。そのデータはボタン1クリックで送信したい先生に送ることができる。また、司会者が全てのデータをcvs形式でダウンロードすることもできる。
\subsection{ファイル構成}
教師画面と学生画面のファイル構成を示す。基本的なファイル構成は同じである。しかし、グラフィックに関するファイル群はインターフェイスに関連するファイルなので内容は異なる。また、細かい部分を見ると、それぞれの内容は異なる。
\subsection{操作手順}
本アプリケーションは教師と学生がインタラクティブにつながることで成り立っている。つまり、教師がアプリケーションを操作したら、即座に学生側の表示も変更される。同様に、学生がアプリケーションを操作したら即座に教師側の画面にも反映される。つまり、教師画面と学生画面を分けて説明することは難しい。そこで、本節では教師画面と学生画面を同時に説明する。
\subsection{待機画面の説明}
はじめに、待機画面の説明を行う。ここで、教師は学生をシステムにログインさせる。学生はXEEのトップページからログインする。ログインしたら待機画面に移動する。すべての学生がログインしたことを確認したら、教師は「次へボタン」をクリックして説明画面に移動する。
\subsection{説明画面の説明}
次に説明画面の説明を行う。ここで、学生は授業評価アンケートの説明を読み、アンケートの回答方法を理解する。すべての学生が説明を読み終えたことを確認したら、教師は「次へボタン」をクリックして「実験」画面に遷移する。
\subsection{実験画面の説明}
次に、実験画面の説明を行う。ここで、学生は複数の質問に回答する。質問は「学年学科の確認」と「授業評価アンケート」の2つに分類される。今回開発した授業評価アンケートでは、「楽しさ」「わかりやすさ」「テストの難しさ」の3つで授業を評価した。この評価はcourse_evaluation_questionnaire/components/EvaluationAxis.js内で静的に定義されている1次元配列のEvaluationAxisを編集することで変更可能である。この変更には、評価軸の追加も含まれる。EvaluationAxisに評価軸の要素を追加することで、自動的にアンケート画面に反映される。評価軸と同様に、質問する教科も外部ファイル内の1次配列から読み込まれている。質問される教科はcourse_evaluation_questionnaire/components/Subjects.js内で静的に定義されている。これも、評価軸と同様に、配列の要素を追加、削除、変更すると、それらがアンケート画面に反映される。

次に、実験画面の説明を行う。はじめに、学生は自分の学年と学科を回答する。これはラジオボタンを用いて実装されている。ここでは、回答ミスを予防するために学年と学科のそれぞれで1つ以上の項目は選択できないように設定している。学年学科を選択したら「次へ」ボタンをクリックする。すると、次の授業評価アンケート画面に遷移する。

授業評価アンケート画面は、以下のような外観になっている。この画面は評価される教科が上から順に並んでいる。学生は、教科ごとのスライダーを動かし1?5の間で授業を評価する。今回開発した授業評価アンケートでは、「楽しさ」「わかりやすさ」「テストの難しさ」の3つの評価軸で授業を評価した。この評価軸は静的配列のEvaluationAxisを編集することで変更可能である。全ての評価軸で授業を評価すると「内容確認画面」に遷移する。ここで「確定」ボタンをクリックすることで、この授業の評価を確定させる。この一連の評価を全ての教科で行うと、ページ最下部にある「NEXT」ボタンがクリック可能になる。NEXTボタンをクリックすると、自分が評価したデータがXEEシステムに送信され、アンケートが終了する。

\subsection{終了画面の説明}
最後に、終了画面の説明をする。名前の通りここはアンケートの終了を通知する画面であるため、学生側には終了画面が表示される。一方で、教師画面では、全学生のアンケート結果をcsv形式で保存することが出来る。
\section{考察}
本節では5段階評価の不正確性について述べ、それに変わる集団の意見集約方法について考察する。溝(2016)は5段階評価システムから得られる調査結果は、合理的かどうかを判断できない。なぜなら、5段階評価システムは個人の回答において評価対象の推移性を保証していないからであると述べている。推移性とは、全ての選択肢において順位付け可能なことである。Sen(1970)は推移性を満たさない選択は全ての選択肢が最適となり得るから、合理的な選択であるか否かを判断しがたいと述べている。つまり、5段階評価システムから得られた回答は合理的な選択であるか不明である。

このことを統計的に確かめるために、ボルダルールを利用した多元的評価システムの開発(溝、2016)で収集されたデータを再集計した。そのデータとは溝(2016)が2016年10月31?11月1日にかけて鹿児島工業高等専門学校3年生5クラス(合計169名)を対象に行った5段階評価システムと多元的評価システムの比較実験のデータである。この論文では、多元的評価システムは評価対象と評価軸にボルダルールを適用して点数算出を行っていた。しかし、これでは真の序列を再現できなかった。そこで、評価軸の重み付けにボルダルールではなく、被験者がAグループとBグループに何倍差があったかという質問で得られた整数を使用した。すると、真の序列を再現することに成功した。

これは、2つのことを示唆している。
1つ目は、5段階評価よりもボルダルールのほうが個人の意見を正確に集約可能である。なぜなら、ボルダルールは全ての比較対象をドラックアンドドロップを用いて順位付けて比較したためである。順位付け可能であるということは推移性を満たしているということである。推移性を満たしていない5段階評価に比べて、推移性を満たしているボルダルールを用いたほうが合理的な回答を期待できる。
2つ目は、その状況に適したスコアリングルールを採用することで結果をより良いものに導ける点である。スコアリングルールとは、社会選択理論への招待を参照。溝(2016)では評価軸のポイント計算においてボルダルールを適用した。その結果、正しい序列は得られなかった。一方、本説で再集計する際には被験者がAグループとBグループの間に感じた点数差を用いたスコアリングルールを適用した。その結果、正しい序列を得ることができた。これは、ボルダルールが要素間の距離を考慮しないために起こったことだと考えられる。ボルダルールでは、最下位から順に1点、2点、3点と得点を付与していき1位には評価対象数だけの得点が与えられる。つまり、評価対象の序列の要素間にそれぞれ異なる距離が設定されていても、それらを無視してしまうということである。例えば、A,B,Cという選手がマラソンで競っているとしよう。AとBはほぼ同着でゴールし1位がAで2位がBであった。一方Cはトップ争いから脱落し、かなり遅れてゴールした。この場合、1位と2位にほとんど差がなく、2位と3位には大きな差がある。それにもかかわらず、ボルダルールで集計すると、Aが3点、Bが2点、Cが1点である。この場合、AとBにはほとんど点数差がなく、BとCには大きな差があると考えるのが自然である。このように、5段階評価よりもボルだルールが優れているからと言って、常にボルだルールを適用すればいいというわけではない。評価するものの特性に合わせて、最適な評価システムを採用することが重要である。そうすることで、より正確に人々の意見を集約することが出来る。

本節をまとめると、以下の2つに集約される。5段階評価システムは推移性を満たしていないため、回答者からの合理的な回答を期待できない。5段階評価システムよりもボルだルールのほうが回答者からの合理的な回答を期待できる。しかし、ボルだルールの万能ではない。質問対象の性質に合ったスコアリングルールを適用することが重要である。
\section{今後の課題}
今回はXEEシステムを使用したため、制約の多いアンケートになった。今後は、試験監督なしで、いつでもどこでも利用できる授業評価アンケートシステムを構築したい。
\section{まとめ}
本研究では、モダンなWEB技術をふんだんに使い授業評価アンケートを構築した。これはXEEシステムをベースに作られている。ユーザーインターフェースはインタラクティブな性質を含ませるためにmaterialーUIを使用した。その結果、わかりやすいアンケートシステムが完成した。
\section{参考文献}
社会選択理論への招待
人間科学 研究法ハンドブック
http://www.geocities.jp/m_hiroi/func/abcerl01.html
\section{謝辞}
%
%
\end{document}
